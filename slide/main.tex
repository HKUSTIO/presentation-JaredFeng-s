\documentclass[aspectratio=169]{beamer}  % 16:9 aspect ratio

% Use a clean theme as base
\usetheme{default}
\usecolortheme{default}

% Custom colors from HKUST logo
\definecolor{hkustblue}{RGB}{0, 51, 119}    % Navy blue from logo
\definecolor{hkustgold}{RGB}{180, 141, 61}  % Golden brown from logo
\definecolor{lightgray}{RGB}{236, 240, 241}

% Customize the appearance
\setbeamercolor{structure}{fg=hkustblue}
\setbeamercolor{background canvas}{bg=white}
\setbeamercolor{normal text}{fg=hkustblue}
\setbeamercolor{frametitle}{fg=hkustblue,bg=white}
\setbeamercolor{itemize item}{fg=hkustgold}
\setbeamercolor{itemize subitem}{fg=hkustgold}
\setbeamercolor{block title}{fg=white,bg=hkustblue}
\setbeamercolor{block body}{fg=hkustblue,bg=lightgray}
\setbeamercolor{title}{fg=hkustblue}
\setbeamercolor{subtitle}{fg=hkustgold}

% Remove navigation symbols
\setbeamertemplate{navigation symbols}{}

% Customize frame title
\setbeamertemplate{frametitle}{
    \vspace*{0.5cm}
    \insertframetitle
    \vspace*{0.2cm}
    \begin{beamercolorbox}[wd=\paperwidth,ht=0.2pt]{structure}
    \end{beamercolorbox}
}

% Customize itemize bullets
\setbeamertemplate{itemize item}{\small\raise0.5pt\hbox{\textbullet}}
\setbeamertemplate{itemize subitem}{\tiny\raise1.5pt\hbox{\textbullet}}

% Packages
\usepackage{graphicx}
\usepackage{amsmath}
\usepackage{hyperref}

% Title page information
\title{Applications of dynamic game model}
\subtitle{By Victor Aguirregabiria, Collard-Wexler, and Stephen P. Ryan}
\author{FENG Huanxi}
\date{\today}

\begin{document}

% Title page
\begin{frame}
    \titlepage
\end{frame}

% Table of contents
\begin{frame}{Outline}
    \item Hospital industry: Gowrisankaran and Town (1997)
    \item    
    \item Cement industry: Ryan (2012) 
    \item    
    \item Hard disk drive industry: Igami \& Uetake (2020)
    \item   
    \item Summary of estimation methodology of dynamic game model
    \item   
    \item Reference
\end{frame}

%section 1
\section{Hospital industry: Gowrisankaran and Town (1997)}
\begin{frame}{Hospital industry: Gowrisankaran and Town (1997)}
    \begin{itemize}
        \item Two types of hospitals (players): for-profit (FP) and non-profit (NP).
        \item Three types of patients in a market: with Medicare (MD,$\Theta_{MD}$), private insurance (PI, $\Theta_{PI}$), and no insurance (UI, $\Theta_{UI}$).
        \item Hospitals makes static decisions (price) and dynamic decisions (investment, exit and entry)
        
    \end{itemize}
\end{frame}

\begin{frame}{Patient's decision}
    \begin{itemize}
        \item  In a market, a fixed population $M$ is ill. So each type of patients has total number: $\Theta_{MD}M$, $\Theta_{PI}M$, $\Theta_{UI}M$. 
        \item Hospital$\ j$ is free to set$\ p_j^{NM}$, the base price that it charges for non-medicare patients.
        \item Poor patients of all medicare types don't have to necessarily pay the full price. The price that each patient must pay is:
        \begin{align*}
        p_{ij}^{MD} &= \max\left\{\min\left(d, Y_{i}^{MD}- Y_{MIN}\right), 0\right\} \\
        p_{ij}^{PI} &= \max\left\{\min\left(cp_{j}^{NM}, Y_{i}^{PI} - Y_{MIN}\right), 0\right\} \\
        p_{ij}^{UI} &= \max\left\{\min\left(p_{i}^{NM}, Y_{i}^{UI} - Y_{MIN}\right), 0\right\}
        \end{align*}
        
    \end{itemize}

\end{frame}

\begin{frame}{Patient's decision}
    \begin{itemize}
        \item  Hence, the utility for patient$\ i$ of type$
        \ T$ choosing hospital $\ j$ is
        \begin{align*}
        U_{ij}^{T} = k_j + \gamma_1 \ln (Y_i^T-\gamma_2 p_{ij}^T)+ \epsilon_{ij}^T  
        \end{align*}
        \item  The utility for patient$\ i$ of type$
        \ T$ buying outside good is
        \begin{align*}
        U_{i0}^{T} = \gamma_1 ln (Y_i^T-\gamma_2 p_{i0}^T)+ \epsilon_{i0}^T  
        \end{align*}
        \item We can obtain the probability (share) of patient$\ i$ choosing hospital $\ j$:
        \begin{align*}
         s_{ij}^{T}(p_{ij}^{T}) &= \frac{\exp\left\{\ k_j + \gamma_1 \ln (\frac{Y_i^T-\gamma_2 p_{ij}^T}{Y_i^T-\gamma_2 p_{i0}^T})\right\}}{1+\sum_{k=1}^J [k_k+\exp(\frac{Y_i^T-\gamma_2 p_{ij}^T}{Y_i^T-\gamma_2 p_{i0}^T})]}
        \end{align*}
    
        
    \end{itemize}

\end{frame}

\begin{frame}{Hospital's static production decision}
    \begin{itemize}
        \item  Each hospital $j$ has identical fixed costs $F$ and differing marginal costs (on each patient) $mc_j=\bar{mc}+bk_j $
        \item Hence, the gross variable profit of $j$ can be constructed as $\pi_j^{GV}(p^{NM})$
        \item Hospitals' static decision is to choose $p_j^{NM}$ to maximize the static profits. 
    
        
    \end{itemize}

\end{frame}

\begin{frame}{Hospital's dynamic decision}
    \begin{itemize}
        \item  State variable vector $s_t=\left\{k_{it}\right\}_{i=1,2..n}$
        \item Timing: in each period, players take actions in this order: 
            \begin{itemize}
                \item Existing hospitals decides whether to exit. Each firm simultaneously receives an i.i.d. scrap value draw from a uniform distribution $U(\phi-\sigma_{\phi},\phi+\sigma_{\phi})$, and immediately decides whether to accept its scrap value draw and exit forever or stay alive in the current period.
                \item Then, the remaining incumbents simultaneously invest. Each hospital $j$ invests amount $x_{jt}$ at a cost of $c_I^{NP}$ or $c_I^{FP}$ .
                \item Next, potential entrants decide whether to enter. There are two potential entrants in each period, one NP and one FP. Each of them receive a sunk cost from distribution $U(S-\sigma_{S},S+\sigma_{S})$. If it enters, it would have an initial quality in the next period.
                \item Finally, a static game starts.
            \end{itemize}
        
    \end{itemize}

\end{frame}

\begin{frame}{Transition between states}
    \begin{itemize}
        \item  Hospital's investment converts to quality in a certain way: \\
        \[
        k_j=g(w_j)
        \]
        \[
         w_{j,t+1}=w_{jt}+v_{jt}-\bar v_t
        \]
        \begin{equation*}  
        v_j=\left\{  
        \begin{array}{lr}  
        1, \text{with prob.  } ax_{jt}/(1+ax_{jt}) \\
        0, \text{with prob.  } 1/(1+ax_{jt})    
        \end{array}  
        \right.  
        \end{equation*}
        
        \begin{equation*}  
        \bar v=\left\{  
        \begin{array}{lr}  
        1, \text{with prob.  } \delta  \\
        0, \text{with prob.  } 1-\delta   
        \end{array}  
        \right.  
        \end{equation*}
        
    \end{itemize}
\end{frame}

\begin{frame}{Estimation method}
   Nested fixed-point algorithm (NFXP)'s central idea:
    \begin{itemize}
        \item Given a structural parameter vector, solve the Bellman equation, get the value function.
        \item Based on the solution, compute the conditional choice probabilities (CCPs) of actions, examine how well these CCPs match observations.
        \item Find a parameter vector that causes the predictions to most closely match the data.
        
    \end{itemize}
\end{frame}
% Section 2
\section{Cement industry: Ryan (2012)}
\begin{frame}{Cement industry: Ryan (2012)}  
    Some features of cement market:
    \begin{itemize}
        \item Cement is a fine mineral dust with binding properties, the key ingredient of concrete. Its production generates significant emissions, making it a frequent target of environmental regulations.
        \item As a concentrated industry, sunk costs of entry and costly investment are important determinants of market structure. 1990 Clean Air Act (CAA) amendment affects the cost structure.
        \item Cement is a largely homogeneous commodity.
        \item Cement easily absorbs water from air, making storage expensive. Short shipping distance makes cement markets quasi-independent geographically.
    \end{itemize}
\end{frame}

\begin{frame}{Model's setup}    
    \begin{itemize}
        \item Model's key setup:
            \begin{itemize}
                \item firm's static decision: quantity of output
                \item firm's dynamic decision: exit or not, enter or not, invest/divest or not
                \item state: firms' capacity in the market. Firms' actions affect the state in the next period. 
            \end{itemize}
        \item In each period, the sequence of events are:
            \begin{itemize}
                \item incumbent firms decide whether to exit the industry
                \item potential entrants and incumbents who decided not to exit simultaneously make entry and investment decisions.
                \item incumbent firms compete over quantities in the market.
                
            \end{itemize}
  
    \end{itemize}
\end{frame}

\begin{frame}{Firms' static decision}    
    \begin{itemize}
    \item Each market is full described by the state vector $s_{t}$. $s_{it}$ is firm $i$'s capacity in $t$.
    \item In each regional market $m$, firms face a constant elasticity of demand curve
    \[
    \ln Q_{m}(\alpha) = \alpha_{0m} + \alpha_1 \ln P_{m},
    \]
    \item the cost of production is:
    \[
    C_i(q_i; \delta) = \delta_0 + \delta_1 q_i + \delta_2 1(q_i > \nu s_i)(q_i - \nu s_i)^2, 
    \]
    where $\delta_0$ is fixed cost, $\delta_1$ is constant marginal cost. If output level $q_{i}$ reaches a threshold $\nu s_i$, it would cause extra costs.
    \item Then, in each period, solve the static Cournot competition, firm $i$'s profit is determined: $\bar{\pi_i}(s;\alpha, \delta)$
  
    \end{itemize}
\end{frame}

\begin{frame}{Firms' investment decision}    
    \begin{itemize}
   \item Recall that firms can adjust their capacity by investment or divestment $x_i$. The cost of adjustments is:
   \[
    \Gamma(x_i;\gamma)=1(x_i>0)(\gamma_{i1}+\gamma_2x_i+\gamma_3x_i^2)+1(x_i<0)(\gamma_{i4}+\gamma_5x_i+\gamma_6x_i^2)
    \]
    if invest, $x_i>0$, fixed cost $\gamma_{1}$ is drawn each period from the common distribution $F_{\gamma}$ $N(\mu_{\gamma}^+,\sigma_{\gamma}^+)$.
    if divest, $x_i<0$, fixed cost $\gamma_{4}$ is drawn each period from the common distribution $G_{\gamma}$ $N(\mu_{\gamma}^-,\sigma_{\gamma}^-)$.
    \end{itemize}
\end{frame}

\begin{frame}{Firms' entry and exit decision}    
    \begin{itemize}
    
   \item Sunk cost of entry and scrap value \\

   \begin{equation}  
    \Phi_i(a_i;k_i,\phi_i)=\left\{  
     \begin{array}{lr}  
        -k_i, &(a_i=enter) \\
        \phi_i, &(a_i=exit)    
     \end{array}  
    \right.  
    \end{equation} 
    fixed cost of entry $k_i$ is private information, drawn from common distribution $F_k$. \\
    $\phi_i$ is the payment firms receive when exiting the market (scape value), drawn from common distribution $F_{\phi}$.
    \item Collecting all the parameters, firm $i$'s per-period payoff function is
    \[
    \pi_i(\mathbf{s}, a; \theta) =\pi_i(\mathbf{s}, a; \alpha, \delta, \gamma_i, k_i, \phi_i) =
    \bar{\pi}_i(\mathbf{s};\alpha, \delta)-\Gamma(x_i;\gamma_i)+ \Phi_i(a_i;k_i,\phi_i)
    \]
    
    \end{itemize}
\end{frame}

\begin{frame}{Transitions between states}    
    \begin{itemize}
    
    \item actions $a_i$ (investment, entry and exit) in $t$ take one period to occur in $s_{t+1}$
    , \\
    such as $s_{i,t+1}=s_{it}+x_{it}$
    
    \end{itemize}
\end{frame}

\begin{frame}{Equilibrium}    
    \begin{itemize}
    \item In a Markovian setting, firms only condition on the current state and their private shocks when making decisions. \\
    Each firm's strategy $a_i=\sigma(s,\epsilon_i)$
    \item Then, we get value functions of incumbents and potential entrants
    \[
    V_i(\mathbf{s}; \sigma(\mathbf{s}), \theta, \epsilon_i) 
    \]
    \item MPNE requires each firm's strategy profile is the best response of competitors' strategy profiles.
    
    \end{itemize}
\end{frame}

\begin{frame}{Estimation method}    
    \begin{itemize}
    \item Use Bajari, Benkard, and Levin (2007)'s two-step estimation method (BBL)
    \item Intuition of BBL:  the econometrician lets the agents in the model solve the dynamic program, and finds parameters of the underlying model such that their behavior is optimal. \\
    
    \end{itemize}
\end{frame}

\begin{frame}{Applications}    
    \begin{itemize}
    \item $\theta= \left\{ \alpha, \delta, \gamma, k, \phi \right\}$
    \item $1^{st}$ step: recover the policy functions governing entry, exit and investment, along with the production relevant parameters \Rightarrow get  
     $ \sigma(s)$, $\alpha$, $\delta$
    \item $2^{nd}$ step: take policy functions and restrictions of MPNE to recover the dynamic parameters \Rightarrow get $ \mu_{\gamma}^+, \sigma_{\gamma}^+, \mu_{\gamma}^-, \sigma_{\gamma}^-, \mu_{\phi}, \sigma_{\phi}, \mu_{k}, \sigma_{k}$
    \end{itemize}
\end{frame}

\begin{frame}{Results}    
    \begin{itemize}
    \item Finding: the CAA increased the mean entry cost by 22 percent, with no significant change on variance. Moreover,  the increase in entry costs greatly reduces the chances that marginal firms enter a market, and this has significant effects on product market competition.
    \item Counterfactual analysis: As a result of entry rates, the overall welfare in the US cement market decreased by at least \$810M.
    \end{itemize}
\end{frame}

% Section 3
\section{Hard disk drive industry: Igami \& Uetake (2020)}
\begin{frame}{Hard disk drive industry: Igami \& Uetake (2020)}
    \begin{itemize}
    \item Antitrust towards mergers is the most important area in which IO economists shape the debate on policy.
    \item Conventional merger analysis takes a proposed merger as given and focuses on its immediate effects on competition. Such a static analysis could be appropriate only if mergers were completely random events. However, \textbf{mergers can be endogenous!}
    
    \end{itemize}
\end{frame}

\begin{frame}{Model's setup}
    \begin{itemize}
    \item Time is discrete with a finite horizon, $t=0, 1, 2...,T$. The final period $T$ is the time at which the demand for HDDs becomes zero (the industry comes to an end).
    \item Finite number of incumbent firms, $i=1,2...,n_t$. Potential entrant $i=0$ \\
    Each firm has productivity on a discretized grid with unit interval $\omega_{it} \in \left\{ \omega^1, \omega^2,...\right\}$, representing its tacit knowledge.
    \item State variable is $\omega_t$ 
    \item A potential entrant exists in every period $a^0 \in A^0=\left\{ enter, out\right\}$. \\
    $a^0$ has sunk cost $k^{a^0}+\epsilon(a_{it}^0)$. \\
    An incumbent take actions in every period $a \in A=\left\{\text{exit}, \text{innovate}, (\text{propose  merger  to j})_{j≠i},(\text{innovate and propose to j})_{j≠i}, idle \right\}$. \\
     $a$ has sunk cost $k^{a}+\epsilon(a_{it})$. \\
   
    \end{itemize}
\end{frame}

\begin{frame}{Transitions between state}
    \begin{itemize}
    \item If incumbent $i$ $a_{it}=exit$, then $\omega_{i,t+1}=\omega^{00}$ ("dead")\\
    \item If $a_{it}=innovate$, then $\omega_{i,t+1}=\omega_{it}+1$ \\
    \item If $a_{it}=\text{propose  merger  to j}$, and firm $j$ takes the deal, then $\omega_{i,t+1}=\text{max}\left\{\omega_{it}, \omega_{jt}\right\}+\triangle_{i, t+1}$. $\triangle_{i, t+1}$ reflects "synergies".
    
    \end{itemize}
\end{frame}

\begin{frame}{Timing}
    \begin{itemize}
    \item Instead of assuming players simultaneous move in each period, here the authors consider an alternating-move game, in which the time interval is relatively short and only one firm can make a dynamic discrete choice within a period. 
    \item The timeline of the stochastically alternating moves:
        \begin{itemize}
            \item At the beginning of each period, nature equally chooses one firm to make a move.
            \item If it's $i$'s turn to make a move, it observes the current industry state $\omega_t$, draws sunk costs of actions.
            \item Based on these information, $i$ makes the discrete choice $a_{it} \in A_{it}$. Corresponding sunk costs immediately occurs. \\
            If $i$ negotiate a merger agreement with $j$, a acquisition price $p_{ij}$ would be paid if the deal is made. It's a take-it-or-leave-it offer.
            \item All active firms participate static competition, earn period profits and pay costs of operation.
        \end{itemize}
    \end{itemize}
\end{frame}

\begin{frame}{Dynamic optimization and equilibrium}
    \begin{itemize}
    \item To reach an incumbent firm $i$'s dynamic optimization, the corresponding Bellman equation is:
    \begin{equation*}
        V_{it} (\omega_t, \epsilon_{it}) = \pi_{it} (\omega_t) - \phi_t (\omega_{it}) + \max \left\{ \bar V^x_{it}, \bar V^c_{it}, \bar V^i_{it}, \bar V^m_{ijt},  {\bar V^{i\&m}_{ijt}} \right\}
    \end{equation*} \\
    $\bar V_{it}^a$ represents conditional values of exiting, idling, innovating, proposing merger to rival $j$, and both merge and innovate. Such as: \\
    \begin{equation*}
        \bar {V}^c_{it} (\omega_t, \epsilon^c_{it}) =  \epsilon^c_{it} + \beta E \left[ V_{i,t+1} (\omega_{t+1}) | \omega_t, a_{it} = idle \right],
    \end{equation*}
    
    \end{itemize}
\end{frame}

\begin{frame}{Estimation method}
    Conditional choice probability (CCP) based method
    \begin{itemize}
    \item $1^{st}$ step: estimate reduced-form CCP function  by using data actions and states.
    \item $2^{nd}$ step: Using these CCP functions, calculate value functions, find optimal structural parameters.
    
    \end{itemize}
\end{frame}

% Section 4
\section{Summary of estimation methodology of dynamic game model}
\begin{frame}{Summary of estimation methodology}
    \begin{itemize}
        \item Full solution methods 
        \begin{itemize}
            \item Nested fixed point algorithm (Gowrisankaran and Town, 1997)
            \item High precision but computationally expensive
        \end{itemize}
        \item BBL (Ryan, 2012)
        \begin{itemize}
            \item Computationally efficient
        \end{itemize}
        
        
        \item CCP (Igami and Uetake, 2020)
        \begin{itemize}
            \item Computationally efficient
        \end{itemize}
    \end{itemize}
\end{frame}

% Section 4
\section{Reference}
\begin{frame}{Reference}
    \begin{itemize}

    \item Gowrisankaran, G., \& Town, R. J. (1997). \textbf{Dynamic Equilibrium in the Hospital Industry.} \emph{Journal of Economics \& Management Strategy}, 6(1), 45-74.

    \item Ryan, S. P. (2012).
    \textbf{The Costs of Environmental Regulation in a Concentrated Industry}.
    \emph{Econometrica}, 80(3), 1019–1061.

    \item Igami, M., and Uetake, K. (2020).
    \textbf{Mergers, Innovation, and Entry-Exit Dynamics: Consolidation of the Hard Disk Drive Industry, 1996–2016}.
    \emph{Review of Economic Studies}, 87(4), 2672–2702.
      
    \end{itemize}
\end{frame}

\end{document} 
